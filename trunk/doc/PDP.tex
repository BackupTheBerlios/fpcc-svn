The PDP comes with three distinct configuration files which have the following purpose:
\begin{itemize}
	\item Diameter peer configuration
	\item PEP settings
	\item Bandwith settings for codecs
\end{itemize}
%%%%%%%%%%%%%%%%%%%%%%%%%%%
\subsubsection{Diameter peer configuration}
As this implementation uses the JavaDiameterPeer implementation from Fraunhofer FOKUS, the peer configuration is done as stated within \cite{website:javaDiameterPeer}.

An example is depicted within Listing \ref{lst:DiamPeerCfg}.

\lstset{language=xml,keywords={DiameterPeer,Acceptor,Auth,Acct,Peer},emph={FQDN, Realm, Vendor_Id, Product_Name, AcceptUnknownPeers, DropUnknownOnDisconnect,id,vendor},keywordstyle=\color{red},emphstyle=\color{blue}}
\begin{lstlisting}[caption=Example of a Diameter Peer configuration file, label={lst:DiamPeerCfg}]
<?xml version="1.0" encoding="UTF-8"?>
<DiameterPeer
		FQDN="pdp.open-ims.test" 
		Realm="open-ims.test" 
		Vendor_Id="10415" 
		Product_Name="JavaDiameterPeer">
	<Acceptor port="3868"/>
	<Auth id="16777216" vendor="10415"/>
	<Auth id="16777216" vendor="0" />
	<Acct id="16777216" vendor="0" />
	<Peer FQDN="pcscf.open-ims.test" Realm="open-ims.test"/>
	<Peer FQDN="pep1.open-ims.test" Realm="open-ims.test"/>
	<Peer FQDN="pep2.open-ims.test" Realm="open-ims.test"/>
	<Auth id="16777236" vendor="10415"></Auth>
	<Acct id="16777236" vendor="10415"></Acct>
	<Auth id="16777236" vendor="0"></Auth>
	<Acct id="16777236" vendor="0"></Acct>
</DiameterPeer>
\end{lstlisting}

All PEPs, which shall be used for enforcement, must be inserted here as Diameter Peers!
See also next section for details.



%%%%%%%%%%%%%%%%%%%%%%%%%%%%%%%%%%%%%%%%%%%%%%%%
\subsubsection{PEP settings}
This file tells the PDP which PEPs it shall use for which user and how much bandwidth each PEP has available.
Listing \ref{lst:PEPContainer} shows an example.

\lstset{language=xml,keywords={PEPContainer,PEP,User},emph={ip,domain,upbw,dobw},keywordstyle=\color{red},emphstyle=\color{blue}}
\begin{lstlisting}[caption=Example of a PEP settings file, label={lst:PEPContainer}]
<?xml version="1.0" encoding="UTF-8"?>
<PEPContainer>
	<PEP ip="10.10.1.1" domain="open-ims.test" upbw="200000" dobw="200000">
	<!--pep1.open-ims.test-->
		<User ip="10.10.2.1"/>
		<User ip="10.10.2.2"/>
		<User ip="10.10.2.3"/>
	</PEP>
	<PEP ip="10.10.1.2" domain="open-ims.test" upbw="200000" dobw="200000">
	<!--pep2.open-ims.test-->
		<User ip="10.10.2.1"/>
		<User ip="10.10.2.2"/>
		<User ip="10.10.2.3"/>
	</PEP>
</PEPContainer>
\end{lstlisting}
This example shows a PDP enforcing at two PEPs at "10.10.1.1" as well as "10.10.1.2" for the users at "10.10.2.X".
Each of them has a total bandwith of 200 KBit per second for all users.

All PEPs, configured within this file have to be in the Diameter peer configuration file as "Peer".
%%%%%%%%%%%%%%%%%%%%%%%%%%%%%%%%%

\subsubsection{Bandwith settings for codecs}
Within the file "codebook.xml", bandwith parameters of used codecs can be set.
Listing \ref{lst:codebook} shows details.

\lstset{language=xml,keywords={codecs,codec},emph={encodingName,payloadType,type,bandwith},keywordstyle=\color{red},emphstyle=\color{blue}}
\begin{lstlisting}[caption=Example of a PEP settings file, label={lst:codebook}]
<?xml version="1.0" encoding="UTF-8"?>
<codecs>
   <codec encodingName="PCMU" payloadType="0" type="audio" bandwidth="87200"/>
   <codec encodingName="GSM" payloadType="3" type="audio" bandwidth="29200"/>
   <codec encodingName="H263" payloadType="34" type="video" bandwith="500000"/>
   <codec encodingName="H263-1998" type="video" bandwidth="500000"/>
</codecs>
\end{lstlisting}
The parameters have the following meaning:
\begin{itemize}
	\item encodingName: the name of the codec as used within the SDP payload
	\item payloadType: default type value as used within SDP (optional)
	\item type: can be audio or video
	\item bandwith: needed bandwith, inclusive all overhead
\end{itemize}



